% Options for packages loaded elsewhere
\PassOptionsToPackage{unicode}{hyperref}
\PassOptionsToPackage{hyphens}{url}
\documentclass[
]{article}
\usepackage{xcolor}
\usepackage{amsmath,amssymb}
\setcounter{secnumdepth}{-\maxdimen} % remove section numbering
\usepackage{iftex}
\ifPDFTeX
  \usepackage[T1]{fontenc}
  \usepackage[utf8]{inputenc}
  \usepackage{textcomp} % provide euro and other symbols
\else % if luatex or xetex
  \usepackage{unicode-math} % this also loads fontspec
  \defaultfontfeatures{Scale=MatchLowercase}
  \defaultfontfeatures[\rmfamily]{Ligatures=TeX,Scale=1}
\fi
\usepackage{lmodern}
\ifPDFTeX\else
  % xetex/luatex font selection
\fi
% Use upquote if available, for straight quotes in verbatim environments
\IfFileExists{upquote.sty}{\usepackage{upquote}}{}
\IfFileExists{microtype.sty}{% use microtype if available
  \usepackage[]{microtype}
  \UseMicrotypeSet[protrusion]{basicmath} % disable protrusion for tt fonts
}{}
\makeatletter
\@ifundefined{KOMAClassName}{% if non-KOMA class
  \IfFileExists{parskip.sty}{%
    \usepackage{parskip}
  }{% else
    \setlength{\parindent}{0pt}
    \setlength{\parskip}{6pt plus 2pt minus 1pt}}
}{% if KOMA class
  \KOMAoptions{parskip=half}}
\makeatother
\setlength{\emergencystretch}{3em} % prevent overfull lines
\providecommand{\tightlist}{%
  \setlength{\itemsep}{0pt}\setlength{\parskip}{0pt}}
\usepackage{bookmark}
\IfFileExists{xurl.sty}{\usepackage{xurl}}{} % add URL line breaks if available
\urlstyle{same}
\hypersetup{
  hidelinks,
  pdfcreator={LaTeX via pandoc}}

\author{}
\date{}

\begin{document}

\textbf{Система}\\
Понятие системы применяется в случаях, когда исследуемый предмет
характеризуется как нечто сложное, о котором невозможно сразу дать
полное определение или графическое представление. Существует множество
форм и аспектов этого понятия.

\textbf{Система} --- это совокупность элементов a(i)a(i)a(i) и связей
или отношений r(j)r(j)r(j).\\
Если не выполняется условие целостности, объект называют компонентом.
Границы между системами и их элементами могут изменяться в зависимости
от подхода к анализу объекта.

\textbf{Связи} играют ключевую роль в обеспечении целостных свойств
системы. С их помощью характеризуются структура системы в статике и ее
функционирование в динамике. Связи ограничивают степень свободы
элементов системы.

Связи классифицируются по нескольким признакам:

\begin{enumerate}
\def\labelenumi{\arabic{enumi}.}
\tightlist
\item
  \textbf{По направленности}:

  \begin{itemize}
  \tightlist
  \item
    Направленные.
  \item
    Ненаправленные.
  \end{itemize}
\item
  \textbf{По силе}:

  \begin{itemize}
  \tightlist
  \item
    Сильные.
  \item
    Слабые.
  \end{itemize}
\item
  \textbf{По типу}:

  \begin{itemize}
  \tightlist
  \item
    Подчинения (например, отношения РОД → ВИД, ЦЕЛОЕ → ЧАСТЬ).
  \item
    Порождения (генетические).
  \item
    Равноправные или безразличные.
  \item
    Управляющие.
  \end{itemize}
\item
  \textbf{По месту приложения}:

  \begin{itemize}
  \tightlist
  \item
    Внутренние.
  \item
    Внешние.
  \end{itemize}
\item
  \textbf{По направленности процессов}:

  \begin{itemize}
  \tightlist
  \item
    Прямые.
  \item
    Обратные.
  \end{itemize}
\end{enumerate}

\textbf{Обратные связи} имеют важное значение в моделировании систем.

\begin{itemize}
\tightlist
\item
  Положительная обратная связь сохраняет тенденцию изменения в системе.
\item
  Отрицательная обратная связь способствует поддержанию требуемого
  значения параметров системы.
\end{itemize}

Обратные связи широко используются в технических устройствах, но их
применение в организационной деятельности может быть сложнее. Для
эффективного использования обратной связи необходимо:

\begin{enumerate}
\def\labelenumi{\arabic{enumi}.}
\tightlist
\item
  Фиксировать рассогласования между требуемыми и текущими параметрами
  системы.
\item
  Учитывать влияние всех элементов системы.
\end{enumerate}

Обратная связь является основой саморегулирования и развития систем. В
сложных саморегулирующихся системах используются одновременно
положительные и отрицательные обратные связи, что обеспечивает их
устойчивость и адаптивность. \#\#\# \textbf{Структуры и формы
представления системы}

Системы можно представлять различными способами в зависимости от уровня
детализации:

\begin{enumerate}
\def\labelenumi{\arabic{enumi}.}
\item
  \textbf{Перечисление элементов}\\
  Простейший способ, где указываются только элементы системы, без
  описания их взаимосвязей.
\item
  \textbf{Модель ``черного ящика''}

  \begin{itemize}
  \tightlist
  \item
    В этой модели отсутствует информация о внутреннем устройстве
    системы.
  \item
    Рассматриваются только входные и выходные связи с окружающей средой.
  \item
    Границы ``черного ящика'' признаются существующими, но их структура
    и механизмы остаются неизвестными.
  \item
    Максимальная формализация ``черного ящика'' включает два множества:

    \begin{itemize}
    \tightlist
    \item
      XXX --- множество входных переменных.
    \item
      YYY --- множество выходных переменных.\\
      При этом взаимосвязи между XXX и YYY не фиксируются (иначе система
      становится ``прозрачным ящиком'').
    \end{itemize}
  \end{itemize}
\end{enumerate}

\textbf{Пример:} Смартфон. Если нас интересует только его функции ввода
(например, голосовые команды) и вывода (звук, изображение), внутренняя
структура устройства может не рассматриваться.

\subsubsection{\texorpdfstring{\textbf{Понятие малого
бизнеса}}{Понятие малого бизнеса}}\label{ux43fux43eux43dux44fux442ux438ux435-ux43cux430ux43bux43eux433ux43e-ux431ux438ux437ux43dux435ux441ux430}

\textbf{Входы} --- это ресурсы, клиенты, внешние условия (например,
погода, климат, солнечная активность), поведение системы и влияние среды
на систему.\\
\textbf{Выходы} --- это влияние системы на среду, которое может быть
выражено через изменения объектов или других элементов окружающего мира.

Особую важность имеет определение цели системы, то есть ее выходов.
Главная цель дополняется связанными целями, которые учитывают
взаимодействие системы с объектами внешней среды.

\begin{center}\rule{0.5\linewidth}{0.5pt}\end{center}

\subsubsection{\texorpdfstring{\textbf{Модель ``черного
ящика''}}{Модель ``черного ящика''}}\label{ux43cux43eux434ux435ux43bux44c-ux447ux435ux440ux43dux43eux433ux43e-ux44fux449ux438ux43aux430}

Модель ``черного ящика'' особенно полезна, а иногда и единственно
возможна при исследовании сложных систем, например:

\begin{itemize}
\tightlist
\item
  \textbf{Психика человека}.
\item
  \textbf{Эффективность лекарств}.
\end{itemize}

В таких случаях доступ к внутреннему устройству системы затруднен, а
выводы делаются исключительно на основании анализа входов и выходов.

Модель ``черного ящика'' применяется, когда требуется изучить систему в
ее обычной среде функционирования. Детализация этой модели приводит к
\textbf{модели состава системы}, где система делится на подсистемы и
элементы:

\begin{itemize}
\tightlist
\item
  \textbf{Система} → \textbf{Подсистема} → \textbf{Элементы}
\end{itemize}

\begin{center}\rule{0.5\linewidth}{0.5pt}\end{center}

\subsubsection{\texorpdfstring{\textbf{Модель состава и структуры
системы}}{Модель состава и структуры системы}}\label{ux43cux43eux434ux435ux43bux44c-ux441ux43eux441ux442ux430ux432ux430-ux438-ux441ux442ux440ux443ux43aux442ux443ux440ux44b-ux441ux438ux441ux442ux435ux43cux44b}

\begin{enumerate}
\def\labelenumi{\arabic{enumi}.}
\item
  \textbf{Модель состава}

  \begin{itemize}
  \tightlist
  \item
    Набор компонентов, из которых состоит система.
  \item
    Показывает подчиненные объекты и их функции.
  \end{itemize}
\item
  \textbf{Модель структуры}

  \begin{itemize}
  \tightlist
  \item
    Множество связей между элементами системы.
  \item
    Отражает организацию системы, подчеркивая устойчивые и значимые
    связи, которые обеспечивают ее существование и функциональность.
  \end{itemize}
\end{enumerate}

\begin{center}\rule{0.5\linewidth}{0.5pt}\end{center}

\subsubsection{\texorpdfstring{\textbf{Структура
системы}}{Структура системы}}\label{ux441ux442ux440ux443ux43aux442ux443ux440ux430-ux441ux438ux441ux442ux435ux43cux44b}

\textbf{Структура} характеризует организацию системы и состоит из связей
между элементами.

\begin{itemize}
\tightlist
\item
  В \textbf{простых системах} структура может быть очевидной и
  неизменной.
\item
  В \textbf{сложных системах} структура отражает наиболее устойчивые
  связи, которые обеспечивают существование системы и ее свойств.
\end{itemize}

\textbf{Структура может быть представлена}:

\begin{itemize}
\tightlist
\item
  Графически (например, диаграммы или схемы).
\item
  В матричной форме (таблицы взаимосвязей).
\item
  С помощью теории множеств или других формальных методов.
\end{itemize}

\begin{center}\rule{0.5\linewidth}{0.5pt}\end{center}

\subsubsection{\texorpdfstring{\textbf{Элементы и
подсистемы}}{Элементы и подсистемы}}\label{ux44dux43bux435ux43cux435ux43dux442ux44b-ux438-ux43fux43eux434ux441ux438ux441ux442ux435ux43cux44b}

\begin{enumerate}
\def\labelenumi{\arabic{enumi}.}
\item
  \textbf{Элемент системы}

  \begin{itemize}
  \tightlist
  \item
    Минимальная часть системы, рассматриваемая как единое целое в рамках
    анализа.
  \item
    Не подлежит дальнейшему делению в данном контексте.
  \end{itemize}
\item
  \textbf{Подсистема}

  \begin{itemize}
  \tightlist
  \item
    Часть системы с относительной целостностью.
  \item
    Подсистемы могут быть:

    \begin{itemize}
    \tightlist
    \item
      \textbf{Основными} --- направленными на достижение общей цели.
    \item
      \textbf{Вспомогательными} --- поддерживающими стабильность и
      эффективность всей системы. \#\#\# \textbf{Структуры системы и их
      особенности}
    \end{itemize}
  \end{itemize}
\end{enumerate}

На разных этапах анализа систему можно представлять с различными
структурами, которые помогают в организации и упрощении работы с ее
сложностью:

\begin{enumerate}
\def\labelenumi{\arabic{enumi}.}
\item
  \textbf{Иерархическая структура}

  \begin{itemize}
  \tightlist
  \item
    Представляет систему в виде уровней подчиненности.
  \item
    Используется для разделения сложного проекта на составные части.
  \item
    Важный аспект --- выделение уровня соподчиненности, который
    определяет отношения между элементами на разных уровнях.
  \item
    Пример: управление организацией или проектом.
  \item
    Если нижний элемент подчиняется двум и более вершинам верхнего
    уровня, это считается структурой со слабыми связями.
  \end{itemize}

  Особые классы иерархий:

  \begin{itemize}
  \tightlist
  \item
    \textbf{СТРАТЫ, слои и эшелоны} --- характеризуются разными
    принципами связанности элементов и различными правами вышележащих
    уровней в отношении нижележащих.
  \end{itemize}

  \textbf{Основная проблема при создании иерархии}: найти компромисс
  между простотой описания, обеспечивающей целостное представление о
  системе, и детализацией, которая отражает ее специфические
  особенности.
\item
  \textbf{Сетевая структура}

  \begin{itemize}
  \tightlist
  \item
    Представляет декомпозицию системы во времени.
  \item
    Используется для отображения последовательности действий, этапов
    деятельности или процесса функционирования.
  \item
    Пример: построение расписания или последовательности технологических
    операций.
  \end{itemize}
\item
  \textbf{Матричная структура}

  \begin{itemize}
  \tightlist
  \item
    Отображает взаимоотношения между смежными уровнями иерархии,
    особенно со слабыми связями.
  \item
    Может быть многомерной, если одна или несколько осей представляют
    собой иерархическую структуру.
  \item
    Пример: \textbf{матричная организационная структура}, которая
    сочетает линейный, функциональный и программно-целевой принципы
    управления.
  \end{itemize}
\item
  \textbf{Структуры с произвольными связями}

  \begin{itemize}
  \tightlist
  \item
    Формируются на начальном этапе анализа.
  \item
    На этом этапе элементы системы делятся, устанавливаются всевозможные
    связи, а затем на основе анализа создается более формализованная
    структура.
  \end{itemize}
\end{enumerate}

\begin{center}\rule{0.5\linewidth}{0.5pt}\end{center}

\subsubsection{\texorpdfstring{\textbf{Классификация
структур}}{Классификация структур}}\label{ux43aux43bux430ux441ux441ux438ux444ux438ux43aux430ux446ux438ux44f-ux441ux442ux440ux443ux43aux442ux443ux440}

\begin{enumerate}
\def\labelenumi{\arabic{enumi}.}
\item
  \textbf{Организационная иерархия}

  \begin{itemize}
  \tightlist
  \item
    Представляет многоцелевую иерархию, где подсистемы образуют
    коалиции.
  \item
    Конфликты между подсистемами решаются вышестоящим эшелоном.
  \end{itemize}
\item
  \textbf{Смешанная иерархия}

  \begin{itemize}
  \tightlist
  \item
    Включает как вертикальные, так и горизонтальные связи.
  \item
    Применяются все виды иерархий, что делает ее гибкой и подходящей для
    сложных систем.
  \end{itemize}
\end{enumerate}

\begin{center}\rule{0.5\linewidth}{0.5pt}\end{center}

\subsubsection{\texorpdfstring{\textbf{Функциональные системы и их
состояние}}{Функциональные системы и их состояние}}\label{ux444ux443ux43dux43aux446ux438ux43eux43dux430ux43bux44cux43dux44bux435-ux441ux438ux441ux442ux435ux43cux44b-ux438-ux438ux445-ux441ux43eux441ux442ux43eux44fux43dux438ux435}

Для функциональной системы ключевым элементом анализа является
\textbf{состояние системы}.\\
Состояние можно определить через:

\begin{itemize}
\tightlist
\item
  Входы и выходы.
\item
  Макропараметры системы (например, энергию, температуру, объем
  ресурсов).
\end{itemize}

\textbf{Примеры состояний системы}:

\begin{itemize}
\tightlist
\item
  Состояние покоя.
\item
  Состояние равномерного прямолинейного движения.
\end{itemize}

Если система способна переходить из одного состояния в другое, говорят,
что она обладает \textbf{поведением}. Анализ поведения системы включает:

\begin{enumerate}
\def\labelenumi{\arabic{enumi}.}
\tightlist
\item
  Определение характера переходов.
\item
  Построение модели, где состояние TTT выражается как функция:
  T=f(Tпредыдущее,Tуправление,Tвлияние)T = f(T\_\{\text{предыдущее}\},
  T\_\{\text{управление}\},
  T\_\{\text{влияние}\})T=f(Tпредыдущее\hspace{0pt},Tуправление\hspace{0pt},Tвлияние\hspace{0pt})
\end{enumerate}

Таким образом, поведение системы зависит от предыдущего состояния,
управляющих воздействий и внешних факторов.

\subsubsection{\texorpdfstring{\textbf{Состояние равновесия и
устойчивости
системы}}{Состояние равновесия и устойчивости системы}}\label{ux441ux43eux441ux442ux43eux44fux43dux438ux435-ux440ux430ux432ux43dux43eux432ux435ux441ux438ux44f-ux438-ux443ux441ux442ux43eux439ux447ux438ux432ux43eux441ux442ux438-ux441ux438ux441ux442ux435ux43cux44b}

\begin{enumerate}
\def\labelenumi{\arabic{enumi}.}
\item
  \textbf{Состояние равновесия}

  \begin{itemize}
  \tightlist
  \item
    Это способность системы при отсутствии внешних воздействий или при
    постоянных влияниях сохранять свое поведение сколько угодно долго.
  \end{itemize}
\item
  \textbf{Устойчивость системы}

  \begin{itemize}
  \tightlist
  \item
    Способность системы возвращаться к состоянию равновесия после
    отклонения под воздействием внешних факторов.
  \item
    Если отклонение небольшое и система способна восстановить
    равновесие, говорят об \textbf{устойчивом равновесии}.
  \item
    Процесс возвращения к равновесию может быть:

    \begin{itemize}
    \tightlist
    \item
      \textbf{Колебательным} (с периодическими изменениями параметров).
    \item
      \textbf{Без колебаний} (плавный возврат).
    \end{itemize}
  \end{itemize}
\end{enumerate}

\begin{center}\rule{0.5\linewidth}{0.5pt}\end{center}

\subsubsection{\texorpdfstring{\textbf{Классификация
систем}}{Классификация систем}}\label{ux43aux43bux430ux441ux441ux438ux444ux438ux43aux430ux446ux438ux44f-ux441ux438ux441ux442ux435ux43c}

Системы можно разделить на классы по различным признакам. Классификация
зависит от задач анализа и способа описания системы.

\paragraph{\texorpdfstring{\textbf{Примеры
классификаций:}}{Примеры классификаций:}}\label{ux43fux440ux438ux43cux435ux440ux44b-ux43aux43bux430ux441ux441ux438ux444ux438ux43aux430ux446ux438ux439}

\begin{enumerate}
\def\labelenumi{\arabic{enumi}.}
\item
  \textbf{По виду отображаемых объектов:}

  \begin{itemize}
  \tightlist
  \item
    Биологические системы.
  \item
    Социальные системы.
  \item
    Технические системы.
  \item
    Экологические системы и др.
  \end{itemize}
\item
  \textbf{По детерминированности:}

  \begin{itemize}
  \tightlist
  \item
    \textbf{Детерминированные} --- поведение системы однозначно
    определяется начальными условиями и воздействиями.
  \item
    \textbf{Стохастические} --- поведение системы имеет элемент
    случайности, который невозможно точно предсказать.
  \end{itemize}
\item
  \textbf{По взаимодействию с внешней средой:}

  \begin{itemize}
  \tightlist
  \item
    \textbf{Открытые системы} --- обмениваются с окружающей средой
    массой, энергией или информацией.

    \begin{itemize}
    \tightlist
    \item
      Частный случай: \textbf{информационно проницаемые системы},
      обменивающиеся только информацией.
    \end{itemize}
  \item
    \textbf{Закрытые системы} --- изолированы от внешней среды.
  \end{itemize}

  \textbf{Пример:}

  \begin{itemize}
  \tightlist
  \item
    Тюрьма --- закрытая система.
  \item
    Офисное здание с интернет-доступом --- открытая информационно
    проницаемая система.
  \end{itemize}
\item
  \textbf{По абстракции:}

  \begin{itemize}
  \tightlist
  \item
    \textbf{Реальные} --- системы, существующие в физической форме.
  \item
    \textbf{Абстрактные} --- теоретические модели систем.
  \end{itemize}
\item
  \textbf{По сложности:}

  \begin{itemize}
  \tightlist
  \item
    \textbf{Простые системы} --- легко описываются и моделируются.
  \item
    \textbf{Сложные системы}:

    \begin{itemize}
    \tightlist
    \item
      \textbf{Большие сложные системы} --- трудно моделировать из-за
      большого числа элементов и связей.
    \item
      \textbf{Сложные системы с недостатком информации} --- модели таких
      систем не дают достаточных данных для эффективного управления.
    \item
      \textbf{Системы с активными элементами} --- системы, где
      элементами выступают люди с индивидуальными ценностями и
      мотивациями.
    \end{itemize}
  \end{itemize}
\end{enumerate}

\begin{center}\rule{0.5\linewidth}{0.5pt}\end{center}

\subsubsection{\texorpdfstring{\textbf{Цели
классификации}}{Цели классификации}}\label{ux446ux435ux43bux438-ux43aux43bux430ux441ux441ux438ux444ux438ux43aux430ux446ux438ux438}

Целью классификации является упрощение анализа систем, выбор подходящих
методов и инструментов для моделирования.

\paragraph{\texorpdfstring{\textbf{Особенности
классификации:}}{Особенности классификации:}}\label{ux43eux441ux43eux431ux435ux43dux43dux43eux441ux442ux438-ux43aux43bux430ux441ux441ux438ux444ux438ux43aux430ux446ux438ux438}

\begin{itemize}
\tightlist
\item
  Системы могут быть отнесены одновременно к нескольким классам, что
  полезно при методах моделирования.
\item
  Классификации всегда относительны:

  \begin{itemize}
  \tightlist
  \item
    В детерминированной системе может быть элемент стохастичности.
  \item
    В закрытой системе могут проявляться черты открытой (например, обмен
    информацией). \#\#\# \textbf{Классификация систем по степени
    организованности}
  \end{itemize}
\end{itemize}

Системы можно классифицировать по степени их организованности, что
позволяет выделить:

\begin{enumerate}
\def\labelenumi{\arabic{enumi}.}
\tightlist
\item
  \textbf{Хорошо организованные системы}.
\item
  \textbf{Плохо организованные системы (диффузные)}.
\item
  \textbf{Самоорганизующиеся системы}.
\end{enumerate}

\begin{center}\rule{0.5\linewidth}{0.5pt}\end{center}

\paragraph{\texorpdfstring{\textbf{Хорошо организованные
системы}}{Хорошо организованные системы}}\label{ux445ux43eux440ux43eux448ux43e-ux43eux440ux433ux430ux43dux438ux437ux43eux432ux430ux43dux43dux44bux435-ux441ux438ux441ux442ux435ux43cux44b}

\begin{itemize}
\item
  Представление объекта или процесса в виде хорошо организованной
  системы предполагает:

  \begin{enumerate}
  \def\labelenumi{\arabic{enumi}.}
  \tightlist
  \item
    Определение элементов системы.
  \item
    Установление связей между элементами.
  \item
    Увязку связей с целями системы.
  \end{enumerate}
\item
  В таких системах \textbf{задача выбора целей и средств их достижения}
  не разделяется. Проблемы выражаются в виде критериев эффективности или
  целевых функций, которые связывают цели со средствами их достижения.
\item
  Для описания применяются \textbf{наиболее существенные элементы и
  связи}.
\item
  Пример:

  \begin{itemize}
  \tightlist
  \item
    При создании модели хорошо организованной системы экспериментально
    подтверждается правомерность детерминированных описаний.
  \end{itemize}
\end{itemize}

\textbf{Ограничения:}

\begin{itemize}
\tightlist
\item
  Хорошо организованные системы неэффективны для описания сложных
  объектов с высокой степенью неопределенности.
\item
  Создание таких моделей требует значительных временных затрат, что
  делает их малоэффективными для решения многокомпонентных задач.
\end{itemize}

\begin{center}\rule{0.5\linewidth}{0.5pt}\end{center}

\paragraph{\texorpdfstring{\textbf{Плохо организованные (диффузные)
системы}}{Плохо организованные (диффузные) системы}}\label{ux43fux43bux43eux445ux43e-ux43eux440ux433ux430ux43dux438ux437ux43eux432ux430ux43dux43dux44bux435-ux434ux438ux444ux444ux443ux437ux43dux44bux435-ux441ux438ux441ux442ux435ux43cux44b}

\begin{itemize}
\tightlist
\item
  В диффузных системах \textbf{задача определения всех компонентов и
  связей не ставится}.
\item
  Характеризуются набором макропараметров и закономерностей, выявленных
  с помощью представительной выборки элементов объекта.
\end{itemize}

\textbf{Особенности:}

\begin{itemize}
\tightlist
\item
  Структурные связи и компоненты могут быть слабо выражены.
\item
  Анализ опирается на выявление закономерностей поведения системы, а не
  на строгую детерминированность.
\end{itemize}

\begin{center}\rule{0.5\linewidth}{0.5pt}\end{center}

\paragraph{\texorpdfstring{\textbf{Самоорганизующиеся
системы}}{Самоорганизующиеся системы}}\label{ux441ux430ux43cux43eux43eux440ux433ux430ux43dux438ux437ux443ux44eux449ux438ux435ux441ux44f-ux441ux438ux441ux442ux435ux43cux44b}

\begin{itemize}
\tightlist
\item
  Эти системы обладают высокой степенью неопределенности, а также
  характеристиками:

  \begin{itemize}
  \tightlist
  \item
    Стохастическое поведение.
  \item
    Нестабильность параметров.
  \end{itemize}
\end{itemize}

\textbf{Основная идея самоорганизации:}\\
Модель развивается и корректируется по мере накопления информации.

\textbf{Этапы построения самоорганизующейся системы:}

\begin{enumerate}
\def\labelenumi{\arabic{enumi}.}
\tightlist
\item
  \textbf{Разработка знаковой системы} для фиксации компонентов и
  связей.
\item
  Преобразование полученного отображения по правилам декомпозиции для
  выявления новых связей.
\item
  \textbf{Накопление информации} о системе.
\item
  \textbf{Повышение адекватности} модели за счет фиксации новых
  элементов и связей.
\end{enumerate}

\textbf{Особенности подхода:}

\begin{itemize}
\tightlist
\item
  Модель становится механизмом развития системы.
\item
  Она может отключаться в стабильных условиях и активироваться при
  изменениях среды.
\item
  Адекватность модели доказывается постепенно, путем оценки корректности
  выделенных компонентов и связей.
\end{itemize}

\begin{center}\rule{0.5\linewidth}{0.5pt}\end{center}

\subsubsection{\texorpdfstring{\textbf{Закономерности функционирования
сложных систем с активными
элементами}}{Закономерности функционирования сложных систем с активными элементами}}\label{ux437ux430ux43aux43eux43dux43eux43cux435ux440ux43dux43eux441ux442ux438-ux444ux443ux43dux43aux446ux438ux43eux43dux438ux440ux43eux432ux430ux43dux438ux44f-ux441ux43bux43eux436ux43dux44bux445-ux441ux438ux441ux442ux435ux43c-ux441-ux430ux43aux442ux438ux432ux43dux44bux43cux438-ux44dux43bux435ux43cux435ux43dux442ux430ux43cux438}

\begin{enumerate}
\def\labelenumi{\arabic{enumi}.}
\tightlist
\item
  \textbf{Целостность или эмерджентность}

  \begin{itemize}
  \tightlist
  \item
    \textbf{Эмерджентность} --- это способность системы проявлять новые
    свойства, которые не присущи отдельным ее компонентам.
  \item
    Эти свойства возникают в результате взаимодействия элементов и
    связей внутри системы.
  \item
    Пример: коллективное поведение в социальном сообществе или работа
    нейронной сети в мозге. \#\#\# \textbf{Метод анализа иерархий (МАИ)}
  \end{itemize}
\end{enumerate}

Метод анализа иерархий позволяет:

\begin{itemize}
\tightlist
\item
  Представить сложные проблемы в виде \textbf{иерархической структуры}.
\item
  Оценить \textbf{приоритеты компонентов проблемы}.
\item
  Выбрать наиболее эффективное решение на основе численных оценок.
\end{itemize}

\begin{center}\rule{0.5\linewidth}{0.5pt}\end{center}

\subsubsection{\texorpdfstring{\textbf{Основные этапы метода анализа
иерархий}}{Основные этапы метода анализа иерархий}}\label{ux43eux441ux43dux43eux432ux43dux44bux435-ux44dux442ux430ux43fux44b-ux43cux435ux442ux43eux434ux430-ux430ux43dux430ux43bux438ux437ux430-ux438ux435ux440ux430ux440ux445ux438ux439}

\begin{enumerate}
\def\labelenumi{\arabic{enumi}.}
\item
  \textbf{Декомпозиция проблемы}

  \begin{itemize}
  \tightlist
  \item
    Проблема разбивается на составляющие элементы, организованные в
    иерархическую структуру:

    \begin{itemize}
    \tightlist
    \item
      Верхний уровень --- \textbf{цель}.
    \item
      Промежуточные уровни --- \textbf{критерии}.
    \item
      Нижний уровень --- \textbf{альтернативы решений}.
    \end{itemize}
  \item
    Пример: Построение иерархии для выбора жилья.
  \end{itemize}
\item
  \textbf{Сравнение элементов}

  \begin{itemize}
  \tightlist
  \item
    Проводится \textbf{попарное сравнение элементов} на каждом уровне
    иерархии.
  \item
    Оценивается \textbf{интенсивность взаимодействия} элементов с
    использованием шкалы Саати (9-балльной шкалы).
  \end{itemize}
\item
  \textbf{Расчет приоритетов}

  \begin{itemize}
  \tightlist
  \item
    На основе матриц попарных сравнений вычисляются \textbf{веса
    критериев} и \textbf{приоритеты альтернатив}.
  \item
    Используются собственные векторы и значения матриц.
  \end{itemize}
\item
  \textbf{Синтез решений}

  \begin{itemize}
  \tightlist
  \item
    На основе вычисленных приоритетов проводится выбор
    \textbf{оптимальной альтернативы}.
  \end{itemize}
\item
  \textbf{Анализ согласованности (ОС)}

  \begin{itemize}
  \tightlist
  \item
    Вычисляется коэффициент согласованности (CI) и отношение
    согласованности (CR).
  \item
    Если CR превышает допустимое значение, оценки необходимо
    пересмотреть.
  \end{itemize}
\end{enumerate}

\begin{center}\rule{0.5\linewidth}{0.5pt}\end{center}

\subsubsection{\texorpdfstring{\textbf{Основные принципы метода
Саати}}{Основные принципы метода Саати}}\label{ux43eux441ux43dux43eux432ux43dux44bux435-ux43fux440ux438ux43dux446ux438ux43fux44b-ux43cux435ux442ux43eux434ux430-ux441ux430ux430ux442ux438}

\begin{enumerate}
\def\labelenumi{\arabic{enumi}.}
\item
  \textbf{Принцип идентичности и декомпозиции}

  \begin{itemize}
  \tightlist
  \item
    Проблема структурируется в виде иерархии или сети.
  \end{itemize}
\item
  \textbf{Принцип дискриминации}

  \begin{itemize}
  \tightlist
  \item
    Обеспечение различимости элементов для оценки.
  \end{itemize}
\item
  \textbf{Принцип сравнительных суждений}

  \begin{itemize}
  \tightlist
  \item
    Попарное сравнение элементов по заданным критериям.
  \end{itemize}
\item
  \textbf{Принцип синтезирования}

  \begin{itemize}
  \tightlist
  \item
    Объединение приоритетов для выбора оптимального решения.
  \end{itemize}
\end{enumerate}

\begin{center}\rule{0.5\linewidth}{0.5pt}\end{center}

\subsubsection{\texorpdfstring{\textbf{Структуры и типы
иерархий}}{Структуры и типы иерархий}}\label{ux441ux442ux440ux443ux43aux442ux443ux440ux44b-ux438-ux442ux438ux43fux44b-ux438ux435ux440ux430ux440ux445ux438ux439}

\begin{enumerate}
\def\labelenumi{\arabic{enumi}.}
\item
  \textbf{Простая иерархия}

  \begin{itemize}
  \tightlist
  \item
    Древовидная структура: цель → критерии → альтернативы.
  \item
    Пример: выбор лучшего проекта.
  \end{itemize}
\item
  \textbf{Китайский ящик (модульная операция)}

  \begin{itemize}
  \tightlist
  \item
    Подразумевает модульное разделение задач с возможностью вложенности.
  \end{itemize}
\item
  \textbf{Полная иерархия}

  \begin{itemize}
  \tightlist
  \item
    Каждый элемент одного уровня связан с каждым элементом уровня ниже.
  \end{itemize}
\end{enumerate}

\begin{center}\rule{0.5\linewidth}{0.5pt}\end{center}

\subsubsection{\texorpdfstring{\textbf{Попарное сравнение и
матрицы}}{Попарное сравнение и матрицы}}\label{ux43fux43eux43fux430ux440ux43dux43eux435-ux441ux440ux430ux432ux43dux435ux43dux438ux435-ux438-ux43cux430ux442ux440ux438ux446ux44b}

\begin{itemize}
\item
  Для оценки элементов строится \textbf{матрица попарных сравнений}:

  Если aija\_\{ij\}aij\hspace{0pt} --- значение предпочтения элемента
  iii над jjj, то aji=1aija\_\{ji\} =
  \frac{1}{a_{ij}}aji\hspace{0pt}=aij\hspace{0pt}1\hspace{0pt}.
\item
  Матрица обратносимметрична. На ее основе:

  \begin{itemize}
  \tightlist
  \item
    Вычисляется \textbf{собственный вектор матрицы} (определяет
    приоритеты).
  \item
    Нормализуется собственный вектор так, чтобы сумма всех значений
    равнялась 1.
  \end{itemize}
\item
  Для проверки согласованности матрицы используется:

  \begin{itemize}
  \item
    \textbf{Индекс согласованности (CI):}

    CI=λmax−nn−1CI =
    \frac{\lambda_{max} - n}{n - 1}CI=n−1λmax\hspace{0pt}−n\hspace{0pt}

    где λmax\lambda\_\{max\}λmax\hspace{0pt} --- максимальное
    собственное значение матрицы, nnn --- размер матрицы.
  \item
    \textbf{Отношение согласованности (CR):}

    CR=CIRICR = \frac{CI}{RI}CR=RICI\hspace{0pt}

    где RIRIRI --- случайный индекс согласованности (зависит от размера
    матрицы).
  \end{itemize}
\item
  Допустимые значения CR:

  \begin{itemize}
  \tightlist
  \item
    Для матрицы 3×3 --- не более 0.05.
  \item
    Для матрицы 5×5 --- не более 0.08.
  \end{itemize}
\end{itemize}

Если значение CR превышает норму, оценки пересматриваются.

\begin{center}\rule{0.5\linewidth}{0.5pt}\end{center}

\subsubsection{\texorpdfstring{\textbf{Преимущества метода анализа
иерархий}}{Преимущества метода анализа иерархий}}\label{ux43fux440ux435ux438ux43cux443ux449ux435ux441ux442ux432ux430-ux43cux435ux442ux43eux434ux430-ux430ux43dux430ux43bux438ux437ux430-ux438ux435ux440ux430ux440ux445ux438ux439}

\begin{itemize}
\tightlist
\item
  Позволяет работать с качественными и количественными данными.
\item
  Обеспечивает структурированный подход к сложным задачам.
\item
  Включает проверку согласованности, что повышает надежность выводов.
  \#\#\# \textbf{Проверка согласованности матриц сравнения}
\end{itemize}

\begin{enumerate}
\def\labelenumi{\arabic{enumi}.}
\item
  \textbf{Нарушение порядка согласованности}

  \begin{itemize}
  \tightlist
  \item
    Сравнения нарушают логику транзитивности:
    a1\textgreater a2,  a2\textgreater a3,  a1\textgreater a3a\_\{1\}
    \textgreater{} a\_\{2\},; a\_\{2\} \textgreater{} a\_\{3\},;
    a\_\{1\} \textgreater{}
    a\_\{3\}a1\hspace{0pt}\textgreater a2\hspace{0pt},a2\hspace{0pt}\textgreater a3\hspace{0pt},a1\hspace{0pt}\textgreater a3\hspace{0pt}
  \item
    Такое нарушение говорит о необходимости пересмотра сравнений.
  \end{itemize}
\item
  \textbf{Нарушение численной согласованности}

  \begin{itemize}
  \tightlist
  \item
    Проверка на числовую согласованность по формуле:
    aij⋅ajk=aika\_\{ij\} \cdot a\_\{jk\} =
    a\_\{ik\}aij\hspace{0pt}⋅ajk\hspace{0pt}=aik\hspace{0pt}
  \item
    Если условие не выполняется, требуется уточнение оценок.
  \end{itemize}
\end{enumerate}

\textbf{Закон иерархической непрерывности}:\\
Элементы нижнего уровня должны быть попарно сравнимы по отношению ко
всем элементам вышестоящего уровня иерархии.

\begin{center}\rule{0.5\linewidth}{0.5pt}\end{center}

\subsubsection{\texorpdfstring{\textbf{Принципы системного анализа
задач}}{Принципы системного анализа задач}}\label{ux43fux440ux438ux43dux446ux438ux43fux44b-ux441ux438ux441ux442ux435ux43cux43dux43eux433ux43e-ux430ux43dux430ux43bux438ux437ux430-ux437ux430ux434ux430ux447}

\begin{enumerate}
\def\labelenumi{\arabic{enumi}.}
\item
  \textbf{Определение связи цели со средствами}

  \begin{itemize}
  \tightlist
  \item
    Если закон связи неизвестен, используются гипотезы, предположения, а
    также системные методы анализа для:

    \begin{itemize}
    \tightlist
    \item
      Формулирования цели.
    \item
      Выявления факторов, влияющих на достижение цели.
    \end{itemize}
  \end{itemize}
\item
  \textbf{Формирование модели}

  \begin{itemize}
  \tightlist
  \item
    Перевод вербального описания в формальное представление:

    \begin{itemize}
    \tightlist
    \item
      \textbf{Гибкость модели}: она должна быть подвержена
      корректировке.
    \item
      Применение специальных приемов: мозговая атака, сценарный анализ,
      экспертные оценки и т.д.
    \end{itemize}
  \end{itemize}
\item
  \textbf{Постепенная формализация задач}

  \begin{itemize}
  \tightlist
  \item
    Постепенная детализация формального описания позволяет:

    \begin{itemize}
    \tightlist
    \item
      Проверить адекватность модели.
    \item
      Уменьшить число итераций для исключения неэффективных вариантов.
    \end{itemize}
  \end{itemize}
\end{enumerate}

\begin{center}\rule{0.5\linewidth}{0.5pt}\end{center}

\subsubsection{\texorpdfstring{\textbf{Классификация методов
моделирования}}{Классификация методов моделирования}}\label{ux43aux43bux430ux441ux441ux438ux444ux438ux43aux430ux446ux438ux44f-ux43cux435ux442ux43eux434ux43eux432-ux43cux43eux434ux435ux43bux438ux440ux43eux432ux430ux43dux438ux44f}

\begin{enumerate}
\def\labelenumi{\arabic{enumi}.}
\item
  \textbf{Методы формализованного представления систем (МФПС)}

  \begin{itemize}
  \tightlist
  \item
    Используются для задач с формализуемыми параметрами.
  \item
    \textbf{Виды методов}:

    \begin{itemize}
    \tightlist
    \item
      \textbf{Аналитические методы}: интегро-дифференциальные уравнения,
      вариационные методы, теория игр, математическое программирование.
    \item
      \textbf{Статистические методы}: теория вероятностей,
      математическая статистика.
    \item
      \textbf{Теоретико-множественные методы}.
    \item
      \textbf{Логические и лингвистические методы}.
    \item
      \textbf{Графические методы}.
    \end{itemize}
  \end{itemize}
\item
  \textbf{Методы активизации интуиции специалистов (МАИС)}

  \begin{itemize}
  \tightlist
  \item
    Направлены на выработку творческих решений:

    \begin{itemize}
    \tightlist
    \item
      Мозговой штурм.
    \item
      Метод Дельфи.
    \item
      Метод дерева целей.
    \item
      Прогнозный граф.
    \end{itemize}
  \end{itemize}
\item
  \textbf{Методы постепенной формализации}

  \begin{itemize}
  \tightlist
  \item
    \textbf{Структурно-лингвистическое моделирование}: используется для
    описания сложных систем с учетом языковой структуры задачи.
  \item
    \textbf{Имитационно-динамическое моделирование}: для исследования
    динамики сложных систем.
  \end{itemize}
\end{enumerate}

\begin{center}\rule{0.5\linewidth}{0.5pt}\end{center}

\subsubsection{\texorpdfstring{\textbf{Особенности применения методов в
экономике}}{Особенности применения методов в экономике}}\label{ux43eux441ux43eux431ux435ux43dux43dux43eux441ux442ux438-ux43fux440ux438ux43cux435ux43dux435ux43dux438ux44f-ux43cux435ux442ux43eux434ux43eux432-ux432-ux44dux43aux43eux43dux43eux43cux438ux43aux435}

\begin{enumerate}
\def\labelenumi{\arabic{enumi}.}
\item
  \textbf{Математическое программирование}

  \begin{itemize}
  \tightlist
  \item
    Оптимизация целевой функции с учетом ограничений.
  \item
    Требования к постановке задачи:

    \begin{itemize}
    \tightlist
    \item
      Указание целевой функции.
    \item
      Определение системы ограничений.
    \end{itemize}
  \end{itemize}
\item
  \textbf{Статистические методы}

  \begin{itemize}
  \tightlist
  \item
    Повышение адекватности модели.
  \item
    Учет разнородных критериев (люди, ресурсы, финансовые ограничения).
  \end{itemize}
\item
  \textbf{Потоковые модели}

  \begin{itemize}
  \tightlist
  \item
    Применяются для управления процессами производства и распределения.
  \end{itemize}
\item
  \textbf{Модели управления запасами}

  \begin{itemize}
  \tightlist
  \item
    Оптимизация количества и частоты пополнения запасов.
  \end{itemize}
\end{enumerate}

\begin{center}\rule{0.5\linewidth}{0.5pt}\end{center}

\subsubsection{\texorpdfstring{\textbf{Календарное
планирование}}{Календарное планирование}}\label{ux43aux430ux43bux435ux43dux434ux430ux440ux43dux43eux435-ux43fux43bux430ux43dux438ux440ux43eux432ux430ux43dux438ux435}

\begin{enumerate}
\def\labelenumi{\arabic{enumi}.}
\item
  \textbf{Методы моделирования}:

  \begin{itemize}
  \tightlist
  \item
    Аналитические.
  \item
    Теоретико-множественные.
  \item
    Лингвистические.
  \item
    Графические.
  \end{itemize}
\item
  \textbf{Особенности планирования}:

  \begin{itemize}
  \tightlist
  \item
    Учет ограничений ресурсов и времени.
  \item
    Оптимизация последовательности выполнения задач.
  \end{itemize}
\end{enumerate}

\subsubsection{\texorpdfstring{\textbf{Классификация методов выбора в
МФПС}}{Классификация методов выбора в МФПС}}\label{ux43aux43bux430ux441ux441ux438ux444ux438ux43aux430ux446ux438ux44f-ux43cux435ux442ux43eux434ux43eux432-ux432ux44bux431ux43eux440ux430-ux432-ux43cux444ux43fux441}

\begin{enumerate}
\def\labelenumi{\arabic{enumi}.}
\item
  \textbf{Методы формализованного представления систем (МФПС)}

  \begin{itemize}
  \tightlist
  \item
    Применяются для анализа и решения проблем, где параметры системы
    можно формализовать.
  \item
    Включают аналитические, статистические, теоретико-множественные и
    другие методы.
  \end{itemize}
\item
  \textbf{Выбор методов по степени организованности системы}

  \begin{itemize}
  \tightlist
  \item
    Организованные системы: для них подходят строго формализованные
    методы.
  \item
    Плохо организованные и самоорганизующиеся системы: требуют гибких и
    менее формализованных подходов.
  \end{itemize}
\end{enumerate}

\subsubsection{\texorpdfstring{\textbf{Методы активизации интуиции
специалистов
(МАИС)}}{Методы активизации интуиции специалистов (МАИС)}}\label{ux43cux435ux442ux43eux434ux44b-ux430ux43aux442ux438ux432ux438ux437ux430ux446ux438ux438-ux438ux43dux442ux443ux438ux446ux438ux438-ux441ux43fux435ux446ux438ux430ux43bux438ux441ux442ux43eux432-ux43cux430ux438ux441}

\begin{enumerate}
\def\labelenumi{\arabic{enumi}.}
\item
  \textbf{Эвристический синтез}

  \begin{itemize}
  \tightlist
  \item
    Ориентирован на усиление качеств экспертного мышления через
    творческие методы.
  \end{itemize}
\item
  \textbf{Классификация эвристических методов}:

  \begin{itemize}
  \item
    \textbf{Методы ненаправленного синтеза}:

    \begin{itemize}
    \tightlist
    \item
      \textbf{Аналогия}: поиск решений в аналогичных областях.
    \item
      \textbf{Инверсия}: изменение восприятия задачи через перестановку
      или переворот.
    \item
      \textbf{Эмпатия}: вовлечение себя в контекст исследуемой системы
      для понимания взаимодействий.
    \item
      \textbf{Идеализация}: моделирование идеального решения, которое
      помогает выявить недочеты в реальных сценариях.
    \end{itemize}
  \item
    \textbf{Методы направленного синтеза}:

    \begin{itemize}
    \tightlist
    \item
      \textbf{Мозговой штурм}: групповая генерация идей.
    \item
      \textbf{Метод гирлянд}: последовательное развитие идей.
    \item
      \textbf{Синептика}: комбинирование разнородных идей для создания
      новых решений.
    \item
      \textbf{Морфологический анализ}: разложение проблемы на элементы и
      их комбинации для поиска решений.
    \end{itemize}
  \end{itemize}
\end{enumerate}

\subsubsection{\texorpdfstring{\textbf{Примеры и активизация творческого
мышления}}{Примеры и активизация творческого мышления}}\label{ux43fux440ux438ux43cux435ux440ux44b-ux438-ux430ux43aux442ux438ux432ux438ux437ux430ux446ux438ux44f-ux442ux432ux43eux440ux447ux435ux441ux43aux43eux433ux43e-ux43cux44bux448ux43bux435ux43dux438ux44f}

\begin{itemize}
\item
  \textbf{Правила для активизации идей}:

  \begin{itemize}
  \tightlist
  \item
    \textbf{Правило 24-х}: фиксация идей, возникающих в течение суток.
  \item
    \textbf{Правило 25-и}: генерация не менее 25 идей для решения
    задачи.
  \item
    \textbf{Правило 26-и}: использование подсказок (например, слова на
    буквы алфавита) для поиска ассоциаций и идей.
  \end{itemize}
\item
  \textbf{Примеры применения}:

  \begin{itemize}
  \tightlist
  \item
    Ситуация с хиппи в США в 60-е годы показывает, как творческие и
    нестандартные подходы могут приводить к неожиданным результатам
    (например, использование ЛСД как культурного феномена).
  \item
    Пример с бизнесом: человек, использующий ведро гнилых яблок как
    элемент декора, чтобы пробудить ностальгию и создать ассоциации.
    \#\#\# \textbf{Основные понятия и принципы}
  \end{itemize}
\item
  \textbf{Противоречия}: фиксируют несоответствия между потребностями
  системы и её возможностями, выявляют области, где необходимо улучшение
  или изменение.
\item
  \textbf{Идеальный конечный результат}: гипотетическое решение, к
  которому следует стремиться. Этот результат используется для отбора
  наиболее подходящих решений, так как лучший вариант будет тот, который
  максимально приближен к идеальному.
\end{itemize}

\subsubsection{\texorpdfstring{\textbf{Классификация методов
эвристического синтеза по ведущему
эффекту}}{Классификация методов эвристического синтеза по ведущему эффекту}}\label{ux43aux43bux430ux441ux441ux438ux444ux438ux43aux430ux446ux438ux44f-ux43cux435ux442ux43eux434ux43eux432-ux44dux432ux440ux438ux441ux442ux438ux447ux435ux441ux43aux43eux433ux43e-ux441ux438ux43dux442ux435ux437ux430-ux43fux43e-ux432ux435ux434ux443ux449ux435ux43cux443-ux44dux444ux444ux435ux43aux442ux443}

\begin{enumerate}
\def\labelenumi{\arabic{enumi}.}
\item
  \textbf{Методы коллективной творческой работы}:

  \begin{itemize}
  \tightlist
  \item
    Основной принцип: коллективное мышление зачастую эффективнее суммы
    индивидуальных решений.
  \item
    Примеры методов: мозговой штурм, конференции идей, синептика, метод
    коллективного блокнота.
  \end{itemize}
\item
  \textbf{Методы анализа комплексных решений}:

  \begin{itemize}
  \tightlist
  \item
    Основной принцип: анализ решений, полученных с помощью разных
    методов.
  \item
    Примеры методов: морфологический анализ и синтез, метод порядочных
    признаков, метод десятичных матриц.
  \end{itemize}
\item
  \textbf{Методы ассоциативного мышления}:

  \begin{itemize}
  \tightlist
  \item
    Основной принцип: использование ассоциаций и аналогий для поиска
    решений.
  \item
    Примеры методов: метод факальных объектов, метод гирлянд случайных
    ассоциаций.
  \end{itemize}
\item
  \textbf{Методы разрешения противоречий}:

  \begin{itemize}
  \tightlist
  \item
    Основной принцип: использование эвристических приемов для поиска
    решений.
  \item
    Примеры методов: алгоритм решения изобретательских задач (АРИЗ),
    библиотека/фонд эвристических приемов.
  \end{itemize}
\end{enumerate}

\subsubsection{\texorpdfstring{\textbf{Методы эвристического
синтеза}}{Методы эвристического синтеза}}\label{ux43cux435ux442ux43eux434ux44b-ux44dux432ux440ux438ux441ux442ux438ux447ux435ux441ux43aux43eux433ux43e-ux441ux438ux43dux442ux435ux437ux430}

\begin{itemize}
\item
  \textbf{Эвристические приемы}: это методы поиска решений, включающие
  творческий подход и интуицию. Примеры: аналогия, инверсия, эмпатия,
  идеализация.
\item
  \textbf{Мозговой штурм}:

  \begin{itemize}
  \tightlist
  \item
    \textbf{Основные правила}: разделение стадии генерации идей и их
    обсуждения. На этапе генерации идей критика запрещена, чтобы не
    прерывать творческий поток.
  \item
    \textbf{Руководитель}: должен стимулировать как можно больше
    вариантов решений.
  \item
    \textbf{Эффективность}: подходит для несложных задач, когда
    участники обладают необходимой информацией. Для сложных задач может
    потребоваться 400-500 идей, что достигается через несколько сессий.
  \end{itemize}
\end{itemize}

\subsubsection{\texorpdfstring{\textbf{Модификации мозгового
штурма}}{Модификации мозгового штурма}}\label{ux43cux43eux434ux438ux444ux438ux43aux430ux446ux438ux438-ux43cux43eux437ux433ux43eux432ux43eux433ux43e-ux448ux442ux443ux440ux43cux430}

\begin{enumerate}
\def\labelenumi{\arabic{enumi}.}
\item
  \textbf{Индивидуальный мозговой штурм}:

  \begin{itemize}
  \tightlist
  \item
    Выполняется одним человеком, который сначала генерирует идеи, а
    затем их оценивает.
  \item
    Длительность сессии: 3-10 минут.
  \end{itemize}
\item
  \textbf{Массовый мозговой штурм}:

  \begin{itemize}
  \tightlist
  \item
    Проводится в большой аудитории, участники делятся на группы по 6-8
    человек.
  \item
    Длительность первого этапа: 15-20 минут.
  \end{itemize}
\item
  \textbf{Двойной мозговой штурм}:

  \begin{itemize}
  \tightlist
  \item
    Совмещает генерацию идей и доброжелательную критику.
  \end{itemize}
\item
  \textbf{Обратный мозговой штурм}:

  \begin{itemize}
  \tightlist
  \item
    Особое внимание уделяется критике идей, что помогает улучшить
    предложения.
  \end{itemize}
\item
  \textbf{Конференции идей (КИ)}:

  \begin{itemize}
  \tightlist
  \item
    Отличаются от мозгового штурма темпом работы и допустимостью
    критики.
  \item
    Участие сотрудников, которые тесно связаны с проблемой, способствует
    генерации уникальных идей.
  \end{itemize}
\item
  \textbf{Конференция 66}:

  \begin{itemize}
  \tightlist
  \item
    Деление большого коллектива на группы по 6 человек для
    мини-конференции, где обсуждают конкретную проблему в течение 6
    минут.
  \end{itemize}
\end{enumerate}

\subsubsection{\texorpdfstring{\textbf{Коллективная генерация идей
(КГИ)}}{Коллективная генерация идей (КГИ)}}\label{ux43aux43eux43bux43bux435ux43aux442ux438ux432ux43dux430ux44f-ux433ux435ux43dux435ux440ux430ux446ux438ux44f-ux438ux434ux435ux439-ux43aux433ux438}

\begin{itemize}
\tightlist
\item
  Метод, направленный на совместное обсуждение и разработку идей группой
  специалистов.
\end{itemize}

\subsubsection{\texorpdfstring{\textbf{Эвристические приемы: примеры и
использование}}{Эвристические приемы: примеры и использование}}\label{ux44dux432ux440ux438ux441ux442ux438ux447ux435ux441ux43aux438ux435-ux43fux440ux438ux435ux43cux44b-ux43fux440ux438ux43cux435ux440ux44b-ux438-ux438ux441ux43fux43eux43bux44cux437ux43eux432ux430ux43dux438ux435}

\begin{itemize}
\tightlist
\item
  \textbf{Метод аналогии}: использование существующих решений из других
  областей для нахождения новых идей.
\item
  \textbf{Метод инверсии}: изменение восприятия задачи через переворот и
  перестановку.
\item
  \textbf{Метод эмпатии}: исследователь погружается в контекст системы,
  чтобы лучше понять её потребности.
\item
  \textbf{Метод идеализации}: моделирование идеального решения для
  нахождения оптимальных реальных решений.
\end{itemize}

\subsubsection{\texorpdfstring{\textbf{Методы ассоциации и
аналогии}}{Методы ассоциации и аналогии}}\label{ux43cux435ux442ux43eux434ux44b-ux430ux441ux441ux43eux446ux438ux430ux446ux438ux438-ux438-ux430ux43dux430ux43bux43eux433ux438ux438}

\begin{itemize}
\tightlist
\item
  \textbf{Метод факальных объектов}: используется перенос признаков
  случайно выбранных объектов на тот, который требуется улучшить.
  Пример: анализ птицы для создания более эффективного летательного
  аппарата.
\item
  \textbf{Метод случайных ассоциаций}: основан на матрице связей между
  объектами и их признаками. В этой матрице элемент Cij=1C\_\{ij\} =
  1Cij\hspace{0pt}=1 указывает на наличие признака jjj у объекта iii;
  Cij=0C\_\{ij\} = 0Cij\hspace{0pt}=0 --- его отсутствие.
\end{itemize}

\subsubsection{\texorpdfstring{\textbf{Синектика}}{Синектика}}\label{ux441ux438ux43dux435ux43aux442ux438ux43aux430}

\begin{itemize}
\tightlist
\item
  Это метод, который помогает находить решения, отказавшись от
  традиционных подходов. Он включает в себя использование интуиции и
  нестандартного мышления, что позволяет выходить за рамки привычного.
\end{itemize}

\subsubsection{\texorpdfstring{\textbf{Метод контроля
вопросов}}{Метод контроля вопросов}}\label{ux43cux435ux442ux43eux434-ux43aux43eux43dux442ux440ux43eux43bux44f-ux432ux43eux43fux440ux43eux441ux43eux432}

\begin{itemize}
\tightlist
\item
  Основной принцип --- задавать направленные вопросы, которые направляют
  исследователя к новому пониманию проблемы. Метод Озборна выделяет 9
  групп вопросов, например:

  \begin{itemize}
  \tightlist
  \item
    Какое новое применение системе можно придумать?
  \item
    Какие возможные модификации функций?
  \item
    Что можно изменить в системе, чтобы улучшить её работу?
  \end{itemize}
\end{itemize}

\subsubsection{\texorpdfstring{\textbf{Метод
сценариев}}{Метод сценариев}}\label{ux43cux435ux442ux43eux434-ux441ux446ux435ux43dux430ux440ux438ux435ux432}

\begin{itemize}
\tightlist
\item
  Этот метод используется для подготовки и согласования представлений о
  проблеме, часто реализуемый с помощью компьютеров. Ранее учитывались
  строгие методики, но с течением времени они были упрощены.
\end{itemize}

\subsubsection{\texorpdfstring{\textbf{Метод экспертных
оценок}}{Метод экспертных оценок}}\label{ux43cux435ux442ux43eux434-ux44dux43aux441ux43fux435ux440ux442ux43dux44bux445-ux43eux446ux435ux43dux43eux43a}

\begin{itemize}
\tightlist
\item
  Предполагается, что неизвестная характеристика исследуемого явления
  является случайной величиной, распределенной в пределах диапазона
  оценок экспертов. Метод применяется, когда есть достаточно информации
  для экспертной оценки, и предполагается, что истинное значение
  находится в пределах этих оценок.
\end{itemize}

\subsubsection{\texorpdfstring{\textbf{Метод
Дельфи}}{Метод Дельфи}}\label{ux43cux435ux442ux43eux434-ux434ux435ux43bux44cux444ux438}

\begin{itemize}
\tightlist
\item
  Разработан Хелмером для снижения влияния психологических факторов в
  группе экспертов. Эксперты получают возможность ознакомиться с
  мнениями и выводами других участников через цикл повторных опросов и
  обратной связи, что помогает выработать более объективные решения.
\end{itemize}

\subsubsection{\texorpdfstring{\textbf{Метод дерева
целей}}{Метод дерева целей}}\label{ux43cux435ux442ux43eux434-ux434ux435ux440ux435ux432ux430-ux446ux435ux43bux435ux439}

\begin{itemize}
\tightlist
\item
  Применяется для представления иерархических структур. Дерево целей
  помогает визуализировать цели и задачи, определяя связи между ними.
  Это также может быть основой для построения дерева решений, что
  упрощает процесс принятия решений.
\end{itemize}

\subsubsection{\texorpdfstring{\textbf{Морфологические
методы}}{Морфологические методы}}\label{ux43cux43eux440ux444ux43eux43bux43eux433ux438ux447ux435ux441ux43aux438ux435-ux43cux435ux442ux43eux434ux44b}

\begin{itemize}
\tightlist
\item
  \textbf{Основной принцип}: структура системы, разработанная Цвикки,
  включает выделение ключевых элементов и их признаков. Затем
  перечисляются возможные значения этих элементов, что позволяет
  генерировать альтернативы через перебор всех сочетаний.
\item
  \textbf{Цели}:

  \begin{enumerate}
  \def\labelenumi{\arabic{enumi}.}
  \tightlist
  \item
    Обеспечение равного интереса ко всем элементам структуры.
  \item
    Ликвидация ограничений для полной структуры.
  \item
    Точная формулировка проблемы.
  \end{enumerate}
\end{itemize}

\subsubsection{\texorpdfstring{\textbf{Другие подходы
Цвикки}:}{Другие подходы Цвикки:}}\label{ux434ux440ux443ux433ux438ux435-ux43fux43eux434ux445ux43eux434ux44b-ux446ux432ux438ux43aux43aux438}

\begin{itemize}
\tightlist
\item
  \textbf{Метод систематического поля (МСП)}: ищет решения на основе
  ограниченного числа опорных знаний, включая факты, законы и
  теоретические положения.
\item
  \textbf{Метод отрицания и конструирования (МОК)}: исследует проблему,
  начиная с предположения её отсутствия и конструирует решения, исходя
  из этого.
\item
  \textbf{Метод морфологического ящика (ММЯ)}: используется для анализа
  всех возможных комбинаций характеристик.
\item
  \textbf{Метод экстремальных ситуаций (МЭС)}: рассматривает наихудшие
  или наилучшие случаи для поиска оптимальных решений.
\item
  \textbf{Метод сопоставления совершенного с дефектным (МССД)}:
  позволяет сравнивать идеальные решения с существующими, чтобы выявить
  их недостатки и способы улучшения.
\item
\end{itemize}

\end{document}
